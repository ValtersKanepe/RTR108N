\documentclass{extbook}
\usepackage[utf8]{inputenc}
\usepackage{amsmath}
\usepackage{fancyhdr}
\usepackage{geometry}

\geometry{
	top=50mm,
	right=50mm,
	left=50mm
}

\newcommand{\resetWhitespace}{
	\setlength{\abovedisplayskip}{4pt}
	\setlength{\belowdisplayskip}{4pt}
}

\pagestyle{fancyplain}
\fancyhf{}
\fancyhead[C]{\small\textit{Theory of information}}
\fancyhead[L]{\small{290}}
\renewcommand{\headrulewidth}{0pt}

\setlength{\headsep}{-10pt}

\begin{document}

\begin{center}

\end{center}

\textbf{Axiom A4} \textit{(Normalization)}.  \textit{H}(1/2, 1/2) = 1.

Axiom A4 means that making a choice from equally probably 
\linebreak
events/outcomes gives one bit of information.

    \textbf{Axiom A5} \textit{(Symmetry)}. The function H(\textit{$p_{1}$}
, \textit{$p_{2}$} , \dots, \textit{$p_{n}$}) is symmetric for any permutation of the arguments \textit{$p_{i}$}.

Axiom A5 means that the funcion H($p_{1}$, $p_{2}$, dots, $p_{n}$) depends only on probabilities $p_{1}$, $p_{2}$, \dots, $p_{n}$ but is independent of their order.

The Axiomatic characterization is obtained in the following theorem.

\textbf{Theorem 3.5.1.} The only funcion that satisfies Axioms A1-A5 is the information entropy function H(\textit{$p_{1}$}
, \textit{$p_{2}$} , \dots, \textit{$p_{n}$}) determined by the formula (3.2.4).

\underline{Proof}. Let us consider the function F(\textit{n}) = \textit{H}(1/\textit{n}, 1/\textit{n}, \dots, 1/\textit{n}) and $r^m$ events with equal probabilites of occurrence. It is possible to decompose choices from all of these events into a series of \textit{m} choices from \textit{r} potential events with equal probobilies of occurrence. In this situation, Axiom A3 gives us the following equality: 

\begin{equation*}
	\resetWhitespace
	\textit{F}(\textit{$r^m$}) = \textit{mF}(\textit{r})
\end{equation*}
Let us fix some number \textit{t}. Then for any natural number \textit{n}, there is number \textit{m} such that
\begin{equation*}
	\resetWhitespace
	\textit{$r^m$} \leq \textit{$t^n$} < \textit{$r^{m+1}$}
	\tag{3.5.1}
\end{equation*}
Indeed, if \textit{m} is the largest number such that \textit{$r^m \leq t^n$} then \textit{$t^n < r^{m+1}$}.
As before, Axiom A3 gives us the following equality: 
\begin{equation}
	\resetWhitespace
	\textit{m}\log_{2}r \leq \textit{n} \log_{2}t < (m+1) \log_{2}\textit{r}
	\tag{3.5.2}
\end{equation}
Dividing all therms in the formula (3.5.2) by \textit{n} $log_{2}$\textit{r}, we have
\begin{equation}
	\resetWhitespace
	\textit{$m/n$} \leq (log_{2}\textit{r}) < \textit{$m/n$} + 1/\textit{n}
	\tag{3.5.3}
\end{equation}
This gives us
\begin{equation}
	\resetWhitespace
	\mid (log_{2}\textit{t}/log_{2}\textit{r}) - \textit{$m/n$}\mid  < 1/\textit{n}
	\tag{3.5.4}
\end{equation}
At the same time, Axiom A2 and inequalities (3.5.1), give us
\begin{equation}
	\resetWhitespace
	\textit{F}(\textit{$r^m$}) \leq \textit{F}(\textit{$t^n$}) \leq \textit{F}(\textit{$r^{m+1}$})
	\notag
\end{equation}

\end{document}
